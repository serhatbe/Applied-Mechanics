% Options for packages loaded elsewhere
\PassOptionsToPackage{unicode}{hyperref}
\PassOptionsToPackage{hyphens}{url}
\PassOptionsToPackage{dvipsnames,svgnames,x11names}{xcolor}
%
\documentclass[
  letterpaper,
  DIV=11,
  numbers=noendperiod]{scrartcl}

\usepackage{amsmath,amssymb}
\usepackage{iftex}
\ifPDFTeX
  \usepackage[T1]{fontenc}
  \usepackage[utf8]{inputenc}
  \usepackage{textcomp} % provide euro and other symbols
\else % if luatex or xetex
  \usepackage{unicode-math}
  \defaultfontfeatures{Scale=MatchLowercase}
  \defaultfontfeatures[\rmfamily]{Ligatures=TeX,Scale=1}
\fi
\usepackage{lmodern}
\ifPDFTeX\else  
    % xetex/luatex font selection
\fi
% Use upquote if available, for straight quotes in verbatim environments
\IfFileExists{upquote.sty}{\usepackage{upquote}}{}
\IfFileExists{microtype.sty}{% use microtype if available
  \usepackage[]{microtype}
  \UseMicrotypeSet[protrusion]{basicmath} % disable protrusion for tt fonts
}{}
\makeatletter
\@ifundefined{KOMAClassName}{% if non-KOMA class
  \IfFileExists{parskip.sty}{%
    \usepackage{parskip}
  }{% else
    \setlength{\parindent}{0pt}
    \setlength{\parskip}{6pt plus 2pt minus 1pt}}
}{% if KOMA class
  \KOMAoptions{parskip=half}}
\makeatother
\usepackage{xcolor}
\setlength{\emergencystretch}{3em} % prevent overfull lines
\setcounter{secnumdepth}{-\maxdimen} % remove section numbering
% Make \paragraph and \subparagraph free-standing
\makeatletter
\ifx\paragraph\undefined\else
  \let\oldparagraph\paragraph
  \renewcommand{\paragraph}{
    \@ifstar
      \xxxParagraphStar
      \xxxParagraphNoStar
  }
  \newcommand{\xxxParagraphStar}[1]{\oldparagraph*{#1}\mbox{}}
  \newcommand{\xxxParagraphNoStar}[1]{\oldparagraph{#1}\mbox{}}
\fi
\ifx\subparagraph\undefined\else
  \let\oldsubparagraph\subparagraph
  \renewcommand{\subparagraph}{
    \@ifstar
      \xxxSubParagraphStar
      \xxxSubParagraphNoStar
  }
  \newcommand{\xxxSubParagraphStar}[1]{\oldsubparagraph*{#1}\mbox{}}
  \newcommand{\xxxSubParagraphNoStar}[1]{\oldsubparagraph{#1}\mbox{}}
\fi
\makeatother


\providecommand{\tightlist}{%
  \setlength{\itemsep}{0pt}\setlength{\parskip}{0pt}}\usepackage{longtable,booktabs,array}
\usepackage{calc} % for calculating minipage widths
% Correct order of tables after \paragraph or \subparagraph
\usepackage{etoolbox}
\makeatletter
\patchcmd\longtable{\par}{\if@noskipsec\mbox{}\fi\par}{}{}
\makeatother
% Allow footnotes in longtable head/foot
\IfFileExists{footnotehyper.sty}{\usepackage{footnotehyper}}{\usepackage{footnote}}
\makesavenoteenv{longtable}
\usepackage{graphicx}
\makeatletter
\newsavebox\pandoc@box
\newcommand*\pandocbounded[1]{% scales image to fit in text height/width
  \sbox\pandoc@box{#1}%
  \Gscale@div\@tempa{\textheight}{\dimexpr\ht\pandoc@box+\dp\pandoc@box\relax}%
  \Gscale@div\@tempb{\linewidth}{\wd\pandoc@box}%
  \ifdim\@tempb\p@<\@tempa\p@\let\@tempa\@tempb\fi% select the smaller of both
  \ifdim\@tempa\p@<\p@\scalebox{\@tempa}{\usebox\pandoc@box}%
  \else\usebox{\pandoc@box}%
  \fi%
}
% Set default figure placement to htbp
\def\fps@figure{htbp}
\makeatother

\KOMAoption{captions}{tableheading}
\usepackage{tikz}
\usepackage{tkz-euclide}
\usetikzlibrary{patterns}
\makeatletter
\@ifpackageloaded{caption}{}{\usepackage{caption}}
\AtBeginDocument{%
\ifdefined\contentsname
  \renewcommand*\contentsname{Table of contents}
\else
  \newcommand\contentsname{Table of contents}
\fi
\ifdefined\listfigurename
  \renewcommand*\listfigurename{List of Figures}
\else
  \newcommand\listfigurename{List of Figures}
\fi
\ifdefined\listtablename
  \renewcommand*\listtablename{List of Tables}
\else
  \newcommand\listtablename{List of Tables}
\fi
\ifdefined\figurename
  \renewcommand*\figurename{Figure}
\else
  \newcommand\figurename{Figure}
\fi
\ifdefined\tablename
  \renewcommand*\tablename{Table}
\else
  \newcommand\tablename{Table}
\fi
}
\@ifpackageloaded{float}{}{\usepackage{float}}
\floatstyle{ruled}
\@ifundefined{c@chapter}{\newfloat{codelisting}{h}{lop}}{\newfloat{codelisting}{h}{lop}[chapter]}
\floatname{codelisting}{Listing}
\newcommand*\listoflistings{\listof{codelisting}{List of Listings}}
\makeatother
\makeatletter
\makeatother
\makeatletter
\@ifpackageloaded{caption}{}{\usepackage{caption}}
\@ifpackageloaded{subcaption}{}{\usepackage{subcaption}}
\makeatother

\usepackage{bookmark}

\IfFileExists{xurl.sty}{\usepackage{xurl}}{} % add URL line breaks if available
\urlstyle{same} % disable monospaced font for URLs
\hypersetup{
  pdftitle={Pendulum},
  colorlinks=true,
  linkcolor={blue},
  filecolor={Maroon},
  citecolor={Blue},
  urlcolor={Blue},
  pdfcreator={LaTeX via pandoc}}


\title{Pendulum}
\author{}
\date{}

\begin{document}
\maketitle


\%\%\%\%\%\%\%\%\%\%\%\%\%\%\%\%\%\%\%\%\%\%\%\%\%\%\%\%\%\%\%\%\%\%\%\%\%\%\%\%\%\%\%\%\%\%\%\%\%\%\%
\% Simple Pendulum \% Author: Gabriel Pereira Coelho \% Email:
gabrielufopalimf@gmail.com \% Publication Date: July 16, 2023 \%
University: Universidade Federal do oeste do Pará \% If you use this
work, please reference it.
\%\%\%\%\%\%\%\%\%\%\%\%\%\%\%\%\%\%\%\%\%\%\%\%\%\%\%\%\%\%\%\%\%\%\%\%\%\%\%\%\%\%\%\%\%\%\%\%\%\%\%

\tikzset{
    support/.style={
        pattern=north east lines,
        draw=none,
        minimum width=0.3,
        minimum height=0.6
    }
    ,>=latex
}
\begin{document}
    \begin{tikzpicture}
       % Coordinates
        \coordinate (A) at (-1.5,0);
        \coordinate (B) at (1.5,0);
        \coordinate (C) at (0,2);
        % Axes
        \draw[->,dashed,gray] (-2,0) -- (2,0);
        \draw[->,dashed, gray] (0,2) -- (0,-2) node[coordinate] (O) {};
        % Support
        \draw (-0.5,2) -- (0.5,2);
        \draw[support] (-0.5,2)  rectangle++ (1,0.25);
        % Coordinate markings (To draw angles and enclose vectors with dashed lines)
        \draw[->] (A) -- ++(270:1.25) node[coordinate] (R) {};
        \draw[->] (A) -- ++(233.2:1) node[coordinate] (P) {};
        \draw[->] (A) -- ++(-36:0.75) node[coordinate] (Q) {};
        % Lines and vectors
        \draw[dashed, gray] (C) -- ++(306.8:2.5);
        \draw (C) -- ++(233.2:2.5);
        \draw[->] (A) -- ++(233.2:1) node[below,rotate=-30]{\tiny{$P_y$}};
        \draw[->] (A) -- ++(53.2:1) node[right,rotate=-30]{\tiny{$T$}};
        \draw[->] (A) -- ++(-36:0.75) node[right,rotate=-30]{\tiny{$P_x$}};
        \draw[->] (A) -- ++(270:1.25) node[below,rotate=10]{\tiny{$P$}};
        \draw[dashed,gray] (P) -- (R) -- (Q);
        % Coordinate markings (To draw angles)
        \draw[->] (A) -- ++(270:1.25) node[coordinate] (R) {};
        \draw[->] (A) -- ++(233.2:1) node[coordinate] (P) {};
        \draw[->] (A) -- ++(-36:0.75) node[coordinate] (Q) {};
        % Angles
        \tkzMarkAngle[-, size=0.5](A,C,O)
        \tkzLabelAngle[pos=0.6](A,C,O){\tiny{$\theta$}}

        \tkzMarkAngle[-, size=0.5](P,A,R)
        \tkzLabelAngle[pos=0.6](P,A,R){\tiny{$\theta$}}
        % Pendulum trajectory
        \draw (-1.5,0) .. controls (-1,-0.5) and (1,-0.5) .. (1.5,0);
        % Circles
        \draw[fill = black] (A) circle (4pt);
        \draw[dashed] (B) circle (4pt); 
    \end{tikzpicture}
\end{document}




\end{document}
