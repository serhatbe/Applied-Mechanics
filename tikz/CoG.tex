\documentclass{article}
\usepackage{amsmath}
\usepackage{geometry}

\geometry{a4paper, margin=1in}

\begin{document}

\section*{Center of Gravity for Common Geometric Shapes}

\subsection*{1. Rectangle or Square}
\[
x_c = \frac{b}{2}, \quad y_c = \frac{h}{2}
\]
- \(b\): Base width  
- \(h\): Height  

The centroid is at the intersection of the diagonals.

\subsection*{2. Triangle}
\[
x_c = \frac{b}{3}, \quad y_c = \frac{h}{3}
\]
(from the base)
- \(b\): Base width  
- \(h\): Height  

The centroid is located \(1/3\) of the distance from the base to the opposite vertex.

\subsection*{3. Circle}
\[
x_c = 0, \quad y_c = 0
\]
- The centroid is at the geometric center of the circle.

\subsection*{4. Semicircle}
\[
x_c = 0, \quad y_c = \frac{4r}{3\pi}
\]
- \(r\): Radius  

The centroid lies along the axis of symmetry, above the flat edge.

\subsection*{5. Quarter-Circle}
\[
x_c = \frac{4r}{3\pi}, \quad y_c = \frac{4r}{3\pi}
\]
- \(r\): Radius  

The centroid lies in the quadrant closer to the circular arc.

\subsection*{6. Trapezoid}
\[
x_c = \frac{b_1 + 2b_2}{3(b_1 + b_2)}, \quad y_c = \frac{h}{2}
\]
- \(b_1, b_2\): Parallel sides  
- \(h\): Height  

The centroid is along the line of symmetry, closer to the larger base.

\subsection*{7. Parabola (Area under y = kx\(^2\))}
\[
x_c = 0, \quad y_c = \frac{3h}{8}
\]
- \(h\): Maximum height  

The centroid lies along the axis of symmetry of the parabola.

\end{document}
